\chapter{Introducción}

En las últimas décadas, las empresas del sector tecnológico han pasado a dominar los mercados bursátiles globales. Compañías como Apple, Amazon, Google, Meta o Microsoft se encuentran entre las de mayor capitalización del mundo, reflejando el auge de la economía digital y basada en activos intangibles. La valuación de empresas tecnológicas se ha convertido así en un tema crucial tanto para inversores como para académicos, especialmente desde episodios históricos como la burbuja \emph{puntocom} de finales de los años 90 hasta las valoraciones récord de \emph{startups} unicornio en tiempos recientes.

Valuar una empresa consiste esencialmente en estimar su valor económico, lo cual sirve para fundamentar decisiones de inversión, procesos de adquisición o fusiones, salidas a bolsa y estrategias corporativas. En el caso de las firmas tecnológicas, este ejercicio presenta desafíos particulares: a menudo operan con modelos de negocio innovadores, crecimiento acelerado, ingresos difíciles de predecir y activos intangibles (como \emph{software}, datos, patentes o marca) que no siempre aparecen reflejados en sus balances financieros tradicionales.

Además, muchas \emph{startups} tecnológicas muestran pérdidas en sus primeros años mientras construyen una base de usuarios o desarrollan productos disruptivos. Todos estos factores hacen que los métodos clásicos de valoración deban adaptarse o complementarse para capturar adecuadamente el valor real de estas compañías.

\section{Objetivos}

El objetivo general de esta investigación es analizar los fundamentos teóricos y metodológicos de la valuación de empresas tecnológicas, identificando las particularidades que distinguen este sector y las tendencias actuales en criterios de valoración en la era digital.

Para alcanzar este propósito principal, se examina el marco teórico general de valoración empresarial y su aplicación específica al sector tecnológico, se identifican y analizan los métodos más utilizados en la valuación de estas empresas, y se caracterizan las particularidades que afectan su valoración. Asimismo, se evalúan los factores cuantitativos y cualitativos que inciden en la valoración de empresas \emph{tech}, se analizan casos de estudio representativos del sector, y se identifican las tendencias actuales que están moldeando los criterios de valoración en el contexto de la transformación digital.

\section{Justificación}

La creciente importancia de las empresas tecnológicas en la economía global y su peso en los mercados financieros hace necesario comprender profundamente los métodos y criterios utilizados para su valoración. Este trabajo contribuye al conocimiento académico y profesional en un área donde los métodos tradicionales de valoración enfrentan desafíos significativos debido a las características particulares del sector tecnológico.

\section{Metodología}

Este trabajo de investigación se basa en una revisión exhaustiva de literatura académica y profesional, análisis de casos de estudio representativos y examen de tendencias actuales en el mercado. Se utiliza un enfoque descriptivo-analítico que combina el marco teórico de valoración financiera con las particularidades del sector tecnológico.

\section{Estructura del trabajo}

En este trabajo de investigación se exploran los fundamentos teóricos de la valuación de empresas, los métodos más utilizados aplicados al sector tecnológico, las características específicas de las empresas tech que afectan su valuación, los factores cuantitativos y cualitativos que inciden en su valor, casos de estudio representativos y las tendencias actuales en criterios de valuación en la era digital.

La estructura sigue un formato académico: primero se presenta el marco teórico general de valoración, luego el desarrollo con el análisis específico del ámbito tecnológico ---incluyendo métodos, factores y casos prácticos---, para finalmente exponer las conclusiones principales y la bibliografía empleada. 
\chapter{Marco Teórico}

La teoría financiera establece que el valor de una empresa se basa en la capacidad de ésta para generar flujos de fondos futuros y en el riesgo asociado a dichos flujos. En términos generales, \textbf{el valor intrínseco de una empresa en marcha es el valor presente de sus flujos de caja futuros esperados, descontados a una tasa que refleje el riesgo de la empresa}.

Este enfoque de \emph{descuento de flujos} considera a la empresa como un ente generador de caja, cuyo patrimonio (acciones) y deuda pueden valorarse al igual que otros activos financieros en función de esos flujos.

\section{Método de Flujo de Caja Descontado (DCF)}

Para aplicar esta metodología, comúnmente denominada \textbf{Flujo de Caja Descontado} (\emph{DCF, Discounted Cash Flow}), es necesario proyectar los flujos de caja libres futuros de la empresa y determinar una tasa de descuento apropiada. Por lo general, se utiliza el costo promedio ponderado de capital o \emph{WACC}, que combina el costo de la deuda y del capital propio.

El resultado es una estimación del valor presente neto de la empresa, considerando su continuidad operacional indefinida. Los métodos basados en DCF son considerados conceptualmente los más correctos para valorar empresas en marcha con perspectivas de continuidad, ya que se fundamentan en principios financieros sólidos (valor tiempo del dinero, relación riesgo-retorno, etc.) \citep{copeland2014,damodaran2012}.

\section{Enfoques principales de valuación}

En la práctica existen diversos enfoques y métodos de valuación que pueden agruparse en tres categorías principales, cada una con sus fundamentos teóricos específicos y aplicaciones apropiadas según el contexto empresarial.

\subsection{Enfoque de ingresos y valor intrínseco}

El enfoque de ingresos incluye el método de descuento de flujos de caja (\emph{DCF}) y sus variantes como el descuento de dividendos o el modelo de valor presente ajustado. Todos ellos comparten la lógica fundamental de calcular el valor presente de beneficios futuros esperados, constituyendo el marco teórico más sólido para la estimación del valor intrínseco empresarial. Este enfoque reconoce que el valor de una empresa deriva de su capacidad de generar flujos de caja libres a sus inversionistas, descontados a una tasa que refleje apropiadamente el riesgo asociado a dichos flujos.

También en esta categoría se inscribe el concepto de \textbf{Valor Económico Agregado} (\emph{EVA}), que mide la utilidad económica residual de la empresa calculada como el beneficio operativo neto menos el costo de oportunidad del capital total invertido. La capitalización a perpetuidad del \emph{EVA} puede utilizarse como método alternativo para valorar la empresa, proporcionando una perspectiva complementaria que enfatiza la creación de valor por encima del costo de capital. Este enfoque resulta particularmente útil para evaluar el desempeño gerencial y la eficiencia en el uso de recursos, aunque presenta limitaciones similares al \emph{DCF} en términos de dependencia de proyecciones futuras.

El método de descuento de flujos constituye el pilar central de la teoría de valuación financiera debido a su fundamentación en principios económicos establecidos. Su fortaleza radica en la consideración explícita del valor tiempo del dinero, la incorporación del riesgo a través de la tasa de descuento, y su capacidad para reflejar las características específicas del negocio mediante las proyecciones de flujos de caja. Sin embargo, su aplicación práctica enfrenta desafíos significativos relacionados con la sensibilidad a supuestos sobre crecimiento futuro, márgenes operativos, inversiones requeridas, y la determinación apropiada de la tasa de descuento.

En empresas tecnológicas, estos desafíos se magnifican debido a la mayor incertidumbre inherente en sus proyecciones financieras, la dificultad de estimar flujos de caja en negocios con modelos emergentes, y la necesidad de capturar el valor de opciones de crecimiento futuro que pueden representar una porción significativa del valor total. Por tanto, aunque el \emph{DCF} mantiene su posición como método de referencia, su aplicación en el contexto tecnológico requiere adaptaciones metodológicas específicas y frecuentemente debe complementarse con enfoques alternativos.

\subsection{Enfoque de mercado y valuación relativa}

El enfoque de mercado comprende los métodos de múltiplos comparables, que valoran la empresa en relación con otras similares mediante la aplicación de indicadores financieros representativos. Los múltiplos más utilizados incluyen \textbf{Precio/Beneficio} (\emph{P/E}), \textbf{Valor Empresa/\emph{EBITDA}}, \textbf{Valor Empresa/Ventas}, y otros específicos del sector. La lógica subyacente consiste en estimar el valor implícito de la empresa si el mercado la valorase de forma análoga a sus pares comparables, asumiendo que empresas similares en términos de riesgo, crecimiento, y rentabilidad deberían comercializarse a múltiplos similares.

Los métodos de múltiplos gozan de gran popularidad en la práctica profesional debido a su simplicidad de aplicación, su capacidad para reflejar las condiciones actuales del mercado, y su utilidad para proporcionar rangos de valoración que pueden contrastarse con resultados obtenidos mediante otros métodos. Además, estos métodos incorporan implícitamente las expectativas y percepciones agregadas del mercado sobre las perspectivas futuras del sector, lo cual puede resultar valioso para capturar tendencias y sentimientos que podrían no estar completamente reflejados en análisis fundamentales individuales.

No obstante, desde una perspectiva teórica estricta, los múltiplos se consideran aproximaciones que deben emplearse con cautela y preferiblemente como complemento del valor obtenido por descuento de flujos. Como señala \cite{fernandez2007}, los múltiplos per se constituyen métodos conceptualmente limitados que asumen eficiencia de mercado y comparabilidad perfecta entre empresas, supuestos que frecuentemente no se cumplen en la realidad. La identificación de empresas verdaderamente comparables representa un desafío particular, especialmente en sectores dinámicos como la tecnología donde las diferencias en modelos de negocio, etapa de desarrollo, y exposición geográfica pueden ser significativas.

En el contexto de empresas tecnológicas, el enfoque de múltiplos presenta ventajas y limitaciones específicas. Por una parte, permite capturar rápidamente el sentimiento del mercado hacia empresas del sector y proporciona referencias útiles para valoraciones en mercados privados donde las transacciones comparables pueden ser escasas. Por otra parte, la volatilidad inherente de los múltiplos tecnológicos, la dificultad de encontrar comparables apropiados para modelos de negocio innovadores, y la tendencia del mercado a valorar expectativas especulativas pueden generar distorsiones significativas en las estimaciones de valor.

\subsection{Métodos especializados y enfoques híbridos}

Además de los enfoques tradicionales, existen métodos especializados particularmente relevantes para la valoración de empresas tecnológicas, incluyendo el enfoque de activos, el método de opciones reales, y técnicas específicas del capital de riesgo. El enfoque de activos considera el valor de los activos netos de la empresa, ya sea en marcha o en caso de liquidación, mediante la evaluación del valor en libros ajustado o el valor de liquidación. Para empresas tecnológicas en crecimiento, este enfoque suele ser menos relevante puesto que el valor de sus activos intangibles fundamentales ---algoritmos, bases de datos, capital humano, relaciones con clientes--- no aparece completamente reflejado en el balance general. Sin embargo, puede proporcionar un piso de valor en casos de empresas maduras o en dificultades financieras.

El método de opciones reales constituye una aplicación de la teoría de opciones financieras a la valoración de empresas o proyectos, reconociendo explícitamente la flexibilidad gerencial y las oportunidades de crecimiento como opciones valiosas. Este método resulta particularmente apropiado para empresas tecnológicas que poseen opciones de expandir proyectos exitosos, entrar en nuevos mercados geográficos o de productos, posponer inversiones hasta obtener información adicional, o abandonar proyectos que no cumplan expectativas. Valorar estas opciones de crecimiento puede revelar valor oculto en empresas con alto potencial pero flujos de caja actuales bajos o negativos, situación común en \emph{startups} tecnológicas.

La aplicación práctica del método de opciones reales enfrenta desafíos técnicos relacionados con la estimación de volatilidades apropiadas, la identificación y valoración de opciones específicas, y la determinación de parámetros del modelo. Sin embargo, proporciona un marco conceptual valioso para comprender por qué el mercado puede asignar valoraciones aparentemente excesivas a empresas tecnológicas jóvenes: gran parte del valor puede residir en opciones futuras más que en flujos de caja inmediatos.

El método \emph{Venture Capital} utilizado por inversores de capital de riesgo para \emph{startups} consiste en estimar un valor futuro de salida ---venta estratégica o salida a bolsa--- y descontarlo a una tasa de retorno objetivo elevada que refleje el riesgo extraordinario, frecuentemente entre 30\% y 50\% anual en etapas tempranas. Es básicamente una variante del \emph{DCF} enfocada en el valor terminal con un horizonte de salida específico, combinada con consideraciones sobre la dilución esperada por futuras rondas de inversión y la probabilidad de éxito o fracaso total.

En resumen, el marco teórico de valuación provee múltiples enfoques complementarios, donde el método de flujo de caja descontado permanece como el pilar central y conceptualmente más sólido para valorar empresas. En la práctica profesional, estos métodos se combinan frecuentemente para proporcionar rangos de valoración que incorporen diferentes perspectivas sobre el valor empresarial. El análisis que sigue en el desarrollo profundiza cómo se aplican y adaptan estos métodos al caso particular de las empresas tecnológicas, y qué características distintivas de estas compañías inciden en los resultados de valuación. 